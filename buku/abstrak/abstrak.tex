% INDONESIAN ABSTRAK
\addcontentsline{toc}{chapter}{ABSTRAK}
\thispagestyle{plain}
\begin{centering}
\textbf{\MakeUppercase{\judul}}
\end{centering}

\begin{tabular}{ll}
Nama  & : \MakeUppercase{\penulis} \\
NRP & : \nrp \\
Departemen  & : \jurusan FTIF-ITS \\
Pembimbing I  & : \pembimbingSatu \\
Pembimbing II  & : \pembimbingDua
\end{tabular}
\\*[20pt]
\begin{centering}
\textbf{Abstrak}
\end{centering}
\itshape
% BEGIN
\\
\indent 
Permasalahan LIS and TREE merupakan sebuah permasalahan yang melibatkan sebuah struktur data tree. Dimana pada tree tersebut akan dicari LIS terpanjang dari seluruh simple path yang ada. Untuk menangani berbagai permasalahan pada permasalahan tersebut dibutuhkan struktur data yang mampu mendukung operasi-operasi tersebut dengan efisien.\\
Pada Tugas Akhir ini akan dirancang penyelesaian permasalahan LIS and TREE antara lain operasi pencarian nilai LIS pada node dan subtree saat ini, operasi update nilai LIS pada node dan subtree saat ini dan menggabungkan serta memindahkan nilai pada dua subtree yang berbeda. Struktur data klasik yang biasa digunakan dalam penyelesaian permasalahan ini merupakan salah satu jenis stuktur data Tree yaitu Segment Tree dengan menggabungkan konsep Disjoint Set Union.\\
Pada Tugas Akhir ini digunakan struktur data Segment Tree dan konsep Disjoint Set Union untuk menyelesaikan operasi-operasi tersebut.
% END
\rm \\
\textbf{Kata Kunci: Disjoint Set Union on Tree, Segment Tree, Longest Increasing Subsequence}


\cleardoublepage

% ENGLISH ABSTRACT
\addcontentsline{toc}{chapter}{ABSTRACT}
\thispagestyle{plain}
\begin{centering}
\textbf{\MakeUppercase{\judulEnglish}}
\end{centering}

\begin{tabular}{ll}
Name  & : \MakeUppercase{\penulis} \\
NRP & : \nrp \\
Major  & : \jurusanEnglish Faculty of IT-ITS \\
Supervisor I  & : \pembimbingSatu \\
Supervisor II  & : \pembimbingDua
\end{tabular}
\\*[20pt]
\begin{centering}
\textbf{Abstract}
\end{centering}
\itshape
% BEGIN
\\
\indent 
LIS and TREE is a problem which involves tree data structure. From that tree a LIS should be found from every simple path that exist. To handle various problem, an efficient data structure is needed to support the operations.\\
This undergraduate thesis will be designed problem solving for LIS and TREE such as operation to find LIS value in a subtree and node, operation to update value of LIS in a subtree and node, and also combine and transfer value of LIS between different subtree and node. Well known data structures, e.g tree specifically segment tree with combination of Disjoint Set Union concept is able to answer the problem efficiently.\\
In this undergraduate thesis Segment Tree data structure and Disjoint Set Union concept will be used to solve those operations.
\\
% END
\rm \\
\textbf{Keywords: Disjoint Set Union on Tree, Segment Tree, Longest Increasing Subsequence}

\cleardoublepage
