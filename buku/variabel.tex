\var{\judul}{Penerapan Konsep DSU on Tree dan Struktur Data Segment Tree Pada Rancang Algoritma: Studi Kasus SPOJ Klasik LIS and TREE}
\var{\judulEnglish}{Implementation of DSU on Tree Concept and Segmnt Tree Data Structure on Algorithm Design on: Case Study in SPOJ Classic Problem LIS and TREEE}
\var{\gelar}{Sarjana Komputer}
\var{\penulis}{Dimas Hirda Pratama}
\var{\nrp}{05111440000147}
\var{\jurusan}{Informatika }
\var{\jurusanEnglish}{Informatics Department }
\var{\fakultas}{Teknologi Informasi dan Komunikasi}
\var{\fakultasEnglish}{Information Technology and Communication}
\var{\prodi}{S-1 }
\var{\bidangStudi}{Algoritma Pemrograman}
\var{\pembimbingSatu}{Rully Soelaiman, S.Kom., M.Kom.}
\var{\nipPembimbingSatu}{197002131994021001}
\var{\pembimbingDua}{Abdul Munif, S.Kom., M.Sc.}
\var{\nipPembimbingDua}{198608232015041004}
\var{\problem}{permasalahan klasik \textit{LIS and TREE}}
\var{\tahun}{2018}
